% ---
%% USPSC-Cap3-Resultados.tex
% --


% --- 
\chapter[Desenvolvimento]{Desenvolvimento}

    \section{Considerações iniciais}
    Este capítulo descreve as etapas de desenvolvimento dos modelos de previsão de chances de indicação e premiação no Oscar, detalhando as escolhas realizadas e os procedimentos adotados no processo. São explicados também os resultados atingidos, os problemas enfrentados e as melhorias que podem ser realizadas em trabalhos futuros.
    
    \section{Atividades realizadas}
    
        \subsection{Obtenção dos dados}
        Foi escolhida como fonte de dados para o trabalho a database The Movies Dataset, da plataforma Kaggle. Trata-se de um conjunto incluindo "metadados para todos os 45.000 filmes listados no conjunto de dados Full MovieLens. O conjunto de dados consiste em filmes lançados em ou antes de julho de 2017. Os dados incluem elenco, equipe, palavras-chave do enredo, orçamento, receita, pôsteres, datas de lançamento, idiomas, empresas de produção, países, contagens de votos TMDB e médias de votos".\cite{kaggle2017}\par
        
        A base de dados The Oscar Award, 1927 - 2020 foi utilizada para complementar os metadados, adicionando a eles as informações de indicações e vitórias na premiação.\cite{kaggle2019}\par

        \subsection{Limpeza e tratamento dos dados}

        \subsection{Arcabouço de análise exploratória}

        \subsection{Criação de modelos}

        \subsection{Avaliação comparativa dos resultados}

        \subsection{Avaliação de importância dos metadados}

        \subsection{Interpretação dos resultados}

    \section{Resultados obtidos}

% --- 

