%% Resumo.tex
% ---
% Resumo
% ---
\setlength{\absparsep}{18pt} % ajusta o espaçamento dos parágrafos do resumo		
\begin{resumo}
	\begin{flushleft} 
			\setlength{\absparsep}{0pt} % ajusta o espaçamento da referência	
			\SingleSpacing 
			\imprimirautorabr~ ~\textbf{\imprimirtitulo}.	\imprimirdata. \pageref{LastPage}p. 
			%Substitua p. por f. quando utilizar oneside em \documentclass
			%\pageref{LastPage}f.
			\imprimirtipotrabalho~-~\imprimirinstituicao, \imprimirlocal, \imprimirdata. 
 	\end{flushleft}
\OnehalfSpacing 			
 Esse trabalho visa a descobrir a aplicabilidade de modelos de aprendizagem de máquina à previsão de chances de indicação e vitória de filmes na premiação dos Academy Awards, ou Oscars. Através de atributos de cada filme, como renda, países de produção, lingua original, entre outros, pretende-se descobrir os melhores modelos para tentar descobrir as chances de cada filme na categoria de Melhor Filme ("Best picture"), e os fatores que influenciam na indicação e premiação.
 

 \textbf{Palavras-chave}: Ciência de Dados, Aprendizagem de Máquina, Cinema, Modelos de Regressão.
\end{resumo}