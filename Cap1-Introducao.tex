%% USPSC-Introducao.tex

% ----------------------------------------------------------
% Introdução (exemplo de capítulo sem numeração, mas presente no Sumário)
% ----------------------------------------------------------
\chapter[Introdução]{Introdução}

A indústria da mineração representa quase 2.5 \% do Produto Interno Bruto brasileiro, tendo movimentado R\$ 209 bilhões em 2020 (GOVERNO DO BRASIL, 2021), e é responsável, entre empregos diretos e indiretos, por 1,9 milhão de postos de trabalho (INSTITUTO BRASILEIRO DE MINERAÇÃO, 2020). A importância e tamanho dessa atividade para o país fazem com que as preocupações com segurança e estabilidade da cadeia produtiva sejam de extrema relevância para o setor e para a sociedade.\par

Dentro das plantas de mineração, as correias transportadoras são um mecanismo chave. Seu funcionamento maximiza a produtividade e velocidade do transporte de minério entre dois pontos. Consequentemente, suas falhas podem ter um custo gigantesco para a atividade (SCHMERSAL, 2020). A correia transportadora é sustentada e guiada por rolos (ou roletes), que serão o objeto deste trabalho (INSTITUTO DE EDUCAÇÃO, CIÊNCIA E TECNOLOGIA DO MARANHÃO, 2009). Ao contrário da correia, são peças intercambiáveis, cuja função principal é proteger e manter o funcionamento da estrutura.\par

Detectar e prever falhas nos rolos é essencial para garantir a continuidade e estabilidade da produção, evitando acidente e a paralisação da produção. Com isso, têm se destacado no setor os rolos inteligentes, também chamados de instrumentados, com sensores de medição e conectados à internet por meio de dispositivos de internet das coisas (IoT). Tais sensores permitem a medição da temperatura, carga, velocidade, e vibração, e da análise cumulativa desses dados. Desta maneira, o objetivo desta pesquisa é descobrir quais medições podem prever, com o máximo de antecedência, falhas e o fim da vida útil de um rolo de transportador.


\cite{governo2021}


	
	
