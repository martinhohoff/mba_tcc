%% USPSC-Introducao.tex

% ----------------------------------------------------------
% Introdução (exemplo de capítulo sem numeração, mas presente no Sumário)
% ----------------------------------------------------------
\chapter[Introdução]{Introdução}

A indústria da mineração representa quase 2.5\% do Produto Interno Bruto brasileiro, tendo movimentado R\$ 209 bilhões em 2020 \cite{governo2021}, e é responsável, entre empregos diretos e indiretos, por 1,9 milhão de postos de trabalho \cite{instituto2020}. A importância e tamanho dessa atividade para o país fazem com que as preocupações com segurança e estabilidade da cadeia produtiva sejam de extrema relevância para o setor e para a sociedade.\par
Dentro das plantas de mineração, as correias transportadoras são um mecanismo chave. Seu funcionamento maximiza a produtividade e velocidade do transporte de minério entre dois pontos. Consequentemente, suas falhas podem ter um custo gigantesco para a atividade (SCHMERSAL, 2020). A correia transportadora é sustentada e guiada por rolos (ou roletes), que serão o objeto deste trabalho (INSTITUTO DE EDUCAÇÃO, CIÊNCIA E TECNOLOGIA DO MARANHÃO, 2009). Ao contrário da correia, são peças intercambiáveis, cuja função principal é proteger e manter o funcionamento da estrutura.\par

Detectar e prever falhas nos rolos é essencial para garantir a continuidade e estabilidade da produção, evitando acidente e a paralisação da produção. Com isso, têm se destacado no setor os rolos inteligentes, também chamados de instrumentados, com sensores de medição e conectados à internet por meio de dispositivos de internet das coisas (IoT). Tais sensores permitem a medição da temperatura, carga, velocidade, e vibração, e da análise cumulativa desses dados. Desta maneira, o objetivo desta pesquisa é descobrir quais medições podem prever, com o máximo de antecedência, falhas e o fim da vida útil de um rolo de transportador.\par

Esse trabalho tem como objetivo principal investigar como as medições de temperatura, velocidade, vibração, e carga em rolos de transporte de minério, além de posição no transportador e fatores externos como umidade e temperatura, se relacionam com a vida útil esperada para os equipamentos. Especificamente, objetiva-se:\par

- detectar quais fatores são os mais preditivos com relação à necessidade de troca ou à possibilidade de falha de um rolo, e com qual antecedência é possível fazer essa previsão;\par
- prever quais os tipos prováveis de falhas, e quais trocas devem ser priorizadas em um mecanismo;\par
- considerar medições que possam complementar a atividade preditiva, como o tipo do minério, o peso da carga, o regime de chuvas, a temperatura média, entre outros;\par
- análise da importância individual dos fatores no que se refere à ocorrência de falhas;\par
- extrapolar as conclusões, se possível, para fatores que influenciam e permitem o cálculo de vida útil em outros equipamentos industriais ligados à mineração.\par
\par
A metodologia para fazer essa investigação seguirá os seguintes passos:\par

1. extração de histórico de dados de rolos instrumentados, considerando as medições já existentes e as ocorrências de falha e/ou fim da vida útil;\par
2. limpeza e tratamento dos dados, verificando-se por dados faltantes, espúrios, e extremos indicados por meio de análises de distribuição;\par
3. criação de um arcabouço de análise exploratória em ambiente Python fazendo-se uso dos pacotes Pandas, Numpy, e Scikit Learn;\par
4. criação de modelos de regressão para o tempo de falha, considerando-se diferentes algoritmos como árvores de decisão, regressão logística, support vector machines, random forest, redes neurais, dentre outros;\par
5. avaliação comparativa dos resultados usando as métricas de mean squared error, root mean squared error, e mean absolute error;\par
6. avaliação da importância de cada fator (medição) baseando-se no feedback dos algoritmos testados;\par
7. interpretação dos resultados para elaboração das conclusões com indicação das técnicas mais promissoras.\par
\par
Entre os resultados esperados, pode-se destacar a descoberta dos melhores modelos de predição de vida útil de rolos que ajudem a gerar eficiência e prevenir acidentes na indústria de mineração e a
obtenção de um ou mais modelos a partir da data de instalação, velocidade, tempo de uso, temperatura, carga e vibração de um rolo, e fatores externos, como temperatura e umidade, para inferir:\par
a) o tempo de vida restante esperado para um rolo;\par
b) o próximo rolo de que se espera uma falha em um mecanismo;\par
c) o tipo de falha esperado;\par
d) interpretação da importância de cada fator (medição) no processo de desgaste e falha dos rolos.\par
