%% USPSC-Introducao.tex

% ----------------------------------------------------------
% Introdução (exemplo de capítulo sem numeração, mas presente no Sumário)
% ----------------------------------------------------------
\chapter[Introdução]{Introdução}
Mesmo sofrendo uma forte queda por conta da pandemia de Covid-19, a indústria de cinema e vídeo gerou uma receita mundial de US\$ 60.4 bilhões em 2021 \cite{indiantelevision2021}. A importância e tamanho dessa atividade fazem das premiações tradicionais um local de destaque para a produção e amostragem de tendências para o setor.\par
Dentro das cerimônias existentes, a premiação da Academia de Artes e Ciências Cinematográficas, popularmente conhecida como "o Oscar", mantém lugar de destaque. Filmes vencedores do Oscar, e até mesmo os indicados, costumam ter seu desempenho de bilheteria alavancado pela premiação. \cite{hdsr2020}.

Detectar e prever tendências de mercado pode ser essencial para aumentar as chances de sucesso de uma produção em premiações, e, consequentemente, o reflexo desse sucesso nas bilheterias. Para isso, podem-se analisar diversos fatores relacionados a cada produção.\par

Esse trabalho tem como objetivo principal investigar como os dados de orçamento, receita, data de lançamento, linguagens, país de produção e empresas produtoras, palavras-chave da trama, créditos de elenco e equipe influenciam nas chances de premiação no Oscar. Especificamente, objetiva-se:\par

- detectar quais fatores são os mais preditivos com relação à chance de nomeação ou vitória de um filme no Oscar, e com qual grau de certeza é possível fazer essa previsão;\par
- prever quais as categorias mais prováveis de premiação;\par
- considerar dados adicionais que possam complementar a atividade preditiva;\par
- análise da importância individual dos fatores nas indicações e premiações;\par
- extrapolar as conclusões, se possível, para fatores que influenciam e permitem o cálculo de chances de indicações e vitórias em outras premiações.\par
\par

A metodologia para fazer essa investigação seguirá os seguintes passos:\par

1. utilização de histórico de metadados de filmes obtidos na plataforma Kaggle;\par
2. limpeza e tratamento dos dados, verificando-se por dados faltantes, espúrios, e extremos indicados por meio de análises de distribuição;\par
3. criação de um arcabouço de análise exploratória em ambiente Python fazendo-se uso dos pacotes Pandas, Numpy, e Scikit Learn;\par
4. criação de modelos de regressão para a chance de premiação, considerando-se diferentes algoritmos como árvores de decisão, regressão logística, support vector machines, random forest, redes neurais, dentre outros;\par
5. avaliação comparativa dos resultados usando as métricas de mean squared error, root mean squared error, e mean absolute error;\par
6. avaliação da importância de cada fator (metadado) baseando-se no feedback dos algoritmos testados;\par
7. interpretação dos resultados para elaboração das conclusões com indicação das técnicas mais promissoras.\par
\par

Entre os resultados esperados, pode-se destacar a descoberta dos melhores modelos de predição de indicações e vitórias que ajudem a gerar previsibilidade do desempenho de um filme em premiações, e a obtenção de um ou mais modelos a partir dos dados de orçamento, receita, data de lançamento, linguagens, país de produção e empresas produtoras, palavras-chave da trama, créditos de elenco e equipe, para inferir:\par

a) a chance de que um filme seja indicado para o Oscar em alguma categoria;\par
b) a categoria (ou categorias) mais prováveis;\par
c) a chance de que o filme, se indicado, ganhe naquela categoria;\par
d) interpretação da importância de cada fator (metadado) nas chances de indicação e premiação.\par
