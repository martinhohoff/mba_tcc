%% USPSC-Cap4-Conclusao.tex
% Capítulo 4 - Conclusão
% ---
% Conclusão
% ---
\chapter[Conclusão]{Conclusão}
% ---
    \section{Interpretação dos resultados}\par
        O primeiro fato de destaque na conclusão dos resultados é que o algoritmo de regressão logística não apresentou bons resultados na predição dos filmes vitoriosos entre os indicados. Isso pode ser devido ao pequeno número de instâncias a serem analisadas (114) ou mesmo à difícil predição do caráter humano da premiação, que é afetada pelos temas de cada filme, tendências artísticas e de conteúdo de cada ano, além de outras nuances sutis. Há que se considerar ainda a hipótese de que um outro modelo, com parâmetros mais bem ajustados, ou outro critério de seleção de atributos, obtenha melhor resultado.
        
        Os resultados ruins do modelo de predição de vitórias contrasta com os bons resultados obtidos na fase anterior, de predição de filmes indicados ao Oscar. Esse modelo de regressão obteve um bom equilíbrio entre precisão e revocação, conseguindo prever um número considerável de verdadeiros positivos, sem incorrer em um grade número de falsos negativos ou falsos positivos.
        
        O modelo de predição de indicações ao Oscar condiz com a expectativa de que fatores conhecidos como data de lançamento, popularidade, média dos votos no site MovieLens, orçamento e renda tivessem grande influência, demonstrando ainda a importância de atributos menos esperados como duração, número de votos no Movie Lens, e se o filme é adulto ou não.
        
        De forma geral, os resultados apontam para a possibilidade de predição das chances de indicação de um filme, desde que conhecidas as variáveis acima. A questão das chances de vitória entre os indicados, por sua vez, permanece em aberto, já que os resultados ruins obtidos não são suficientes para concluir a impossibilidade dessa predição.

    \section[Dificuldades encontradas]{Dificuldades encontradas}

        Alguns dos problemas encontrados disseram respeito ao trabalho com dois datasets separados, um contendo dados da premiação do Oscar e outro com metadados dos filmes. Boa parte do esforço inicial do projeto se concentrou em formas de unir esses dois datasets. Mesmo assim, alguns dos filmes do dataset do Oscar não foram encontrados no dataset de metadados, e por isso foram deixados de fora. É razoável supor que o inverso aconteceu, e filmes no dataset de metadados não tiveram seus dados de premiação adicionados durante a integração de esquemas.
        
        A imprecisão de anos e nomes dos filmes, que utilizava formatos diferentes em cada dataset, e até mesmo inconsistentes dentro de um mesmo dataset, geraram grande dificuldade de associação entre os dados de um dataset e de outro. Por isso, a melhor hipótese talvez seja a criação e aprimoração de um dataset próprio para essa análise.
        
        No campo da metodologia, um problema recorrente ao longo da pesquisa foi o de definição do problema. Eventualmente, notamos que se tratava de pelo menos dois problemas - chances de indicação e de premiação -, com a grande diferença de que os dados de indicação são classe alvo no primeiro problema e atributos no segundo problema.
        
        Como as categorias não são mutuamente excludentes, pretendíamos ainda dividir cada um desses 2 problemas em 23, que é o número de categorias do Oscar, tentando analisar as chances de um filme em cada uma dessas categorias. O tamanho desse problema revelou-se excessivamente desafiador para essa pesquisa, que preferiu manter-se nas chances de indicação e premiação em uma única categoria (Oscar de melhor filme).
    
    \section[Próximas etapas da pesquisa]{Próximas etapas}
        Um subproduto possível da pesquisa é a criação do dataset unificado de metadados e indicações ao Oscar. A validação e consolidação desse dataset, ou a utilização de um dataset de metadados que já traga informações da premiação do Oscar, poderia gerar resultados melhores, e concentraria o esforço da pesquisa na possibilidade ou não de prever indicações, e com quais modelos.
        
        Próximos passos possíveis para a pesquisa também incluem a análise da predição com outros algoritmos além da regressão logística, e com expansão do problema para as outras categorias além de melhor filme.
